\documentclass{classes/relatorio}

% ==================================================
% CONFIGURAÇÕES DO CABEÇALHO E RODAPÉ
% ==================================================
\instituicao{Nome da Instituição}
\curso{Nome do Curso}
\disciplina{Nome da Disciplina}
\professor{Prof.(a) Nome do Professor}
\aluno{Nome do Aluno}
\titulo{Título do Relatório}

% ==================================================
% INÍCIO DO DOCUMENTO
% ==================================================
\begin{document}

% ==================================================
% INTRODUÇÃO
% ==================================================
\section{Introdução}

A introdução deve apresentar o tema do relatório de forma clara e objetiva, contextualizando o leitor sobre o assunto que será abordado. É importante que sejam estabelecidos os objetivos do trabalho e a justificativa para sua realização.

Este documento foi estruturado seguindo as normas da ABNT (Associação Brasileira de Normas Técnicas) para trabalhos acadêmicos, garantindo padronização e qualidade na apresentação dos resultados.

\subsection{Objetivos}

\subsubsection{Objetivo Geral}

Apresentar um modelo de relatório técnico formatado segundo as normas ABNT, facilitando a elaboração de documentos acadêmicos.

\subsubsection{Objetivos Específicos}

\begin{itemize}
    \item Demonstrar a estrutura básica de um relatório;
    \item Apresentar exemplos de citações e referências;
    \item Ilustrar o uso de figuras, tabelas e códigos.
\end{itemize}

% ==================================================
% DESENVOLVIMENTO
% ==================================================
\section{Desenvolvimento}

O desenvolvimento é a parte principal do relatório, onde são apresentados os dados, análises, resultados e discussões pertinentes ao tema abordado.

\subsection{Fundamentação Teórica}

Segundo \citeonline{knuth1984texbook}, a composição tipográfica de documentos técnicos requer atenção especial aos detalhes de formatação e estruturação do texto.

\citeauthor{lamport1994latex} (\citeyear{lamport1994latex}) destaca a importância de sistemas de preparação de documentos que permitam ao autor focar no conteúdo, enquanto o sistema cuida da formatação.

\subsection{Metodologia}

A metodologia descreve os procedimentos utilizados para a realização do trabalho. Pode incluir descrição de experimentos, coleta de dados, ferramentas utilizadas, entre outros aspectos relevantes.

\subsection{Resultados}

\subsubsection{Figuras}

As figuras devem ser inseridas no texto com legendas descritivas, conforme exemplo da Figura \ref{fig:exemplo}.

\begin{figure}[htbp]
    \centering
    % \includegraphics[width=0.6\textwidth]{img/exemplo.png}
    \fbox{\parbox{0.6\textwidth}{\centering [Insira sua imagem aqui]\\img/exemplo.png}}
    \caption{Exemplo de legenda de figura}
    \label{fig:exemplo}
\end{figure}

\subsubsection{Tabelas}

As tabelas seguem o padrão ABNT, com bordas superiores e inferiores, conforme Tabela \ref{tab:exemplo}.

\begin{table}[htbp]
    \centering
    \caption{Exemplo de tabela formatada segundo ABNT}
    \label{tab:exemplo}
    \begin{tabular}{lcc}
        \hline
        \textbf{Item} & \textbf{Valor 1} & \textbf{Valor 2} \\
        \hline
        Linha 1 & 10,5 & 20,3 \\
        Linha 2 & 15,7 & 18,9 \\
        Linha 3 & 12,3 & 22,1 \\
        \hline
    \end{tabular}
\end{table}

\subsubsection{Códigos}

Códigos-fonte podem ser inseridos diretamente ou importados de arquivos externos, como demonstrado no Código \ref{cod:python}.

\lstinputlisting[language=Python, caption=Exemplo de código Python, label=cod:python]{cod/exemplo.py}

Alternativamente, códigos curtos podem ser inseridos inline:

\begin{lstlisting}[language=Python, caption=Código inline]
def funcao_exemplo(x, y):
    return x + y

resultado = funcao_exemplo(5, 3)
print(f"Resultado: {resultado}")
\end{lstlisting}

% ==================================================
% CONCLUSÃO
% ==================================================
\section{Conclusão}

A conclusão deve retomar os objetivos apresentados na introdução e sintetizar os principais resultados obtidos no desenvolvimento do trabalho. É importante que seja concisa e objetiva, destacando as contribuições do estudo e possíveis trabalhos futuros.

Este modelo de relatório demonstrou a aplicação das normas ABNT para documentos técnicos, fornecendo exemplos práticos de estruturação, citações, figuras, tabelas e códigos-fonte.

% ==================================================
% REFERÊNCIAS
% ==================================================
\bibliography{referencias}

\end{document}
