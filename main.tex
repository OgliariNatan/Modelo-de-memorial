\newif\ifminuta
\minutatrue
%\minutafalse
\documentclass{classes/relatorio}

% ==================================================
% CONFIGURAÇÕES DO CABEÇALHO E RODAPÉ
% ==================================================
\logo{img/logo.png}  % Caminho para a logo
\superior{ENGENHEIRO DE SOFTWARE}
\curso{TÉCNICO EM ELETROELETRÔNICA}
\nome{NATAN OGLIARI}

\titulo{Memorial Descritivo}

% ==================================================
% INÍCIO DO DOCUMENTO
% ==================================================
\begin{document}

% ==================================================
% INTRODUÇÃO
% ==================================================


\section{Introdução}

A introdução deve apresentar o tema do relatório de forma clara e objetiva, contextualizando o leitor sobre o assunto que será abordado. É importante que sejam estabelecidos os objetivos do trabalho e a justificativa para sua realização.

Este documento foi estruturado seguindo as normas da ABNT (Associação Brasileira de Normas Técnicas) para trabalhos acadêmicos, garantindo padronização e qualidade na apresentação dos resultados.

\subsection{Objetivos}

\subsubsection{Objetivo Geral}

Apresentar um modelo de relatório técnico formatado segundo as normas ABNT, facilitando a elaboração de documentos acadêmicos.

\subsubsection{Objetivos Específicos}

\begin{itemize}
    \item Demonstrar a estrutura básica de um relatório;
    \item Apresentar exemplos de citações e referências;
    \item Ilustrar o uso de figuras, tabelas e códigos.
\end{itemize}

% ==================================================
% DESENVOLVIMENTO
% ==================================================
\section{Desenvolvimento}

O desenvolvimento é a parte principal do relatório, onde são apresentados os dados, análises, resultados e discussões pertinentes ao tema abordado.

\subsection{Fundamentação Teórica}

A composição tipográfica de documentos técnicos requer atenção especial aos detalhes de formatação e estruturação do texto.

Destaca a importância de sistemas de preparação de documentos que permitam ao autor focar no conteúdo, enquanto o sistema cuida da formatação.

\subsection{Metodologia}

A metodologia descreve os procedimentos utilizados para a realização do trabalho. Pode incluir descrição de experimentos, coleta de dados, ferramentas utilizadas, entre outros aspectos relevantes.

\subsection{Resultados}

\subsubsection{Figuras}

As figuras devem ser inseridas no texto com legendas descritivas, conforme exemplo da Figura \ref{fig:exemplo}.

\begin{figure}[h]
    \centering
    \caption{Exemplo de legenda de figura}
    \includegraphics[width=0.6\textwidth]{img/logo.png}
    \label{fig:exemplo}

    {\fontsize{10pt}{\baselineskip}\selectfont
   Fonte: O autor (2026)}
\end{figure}

\subsubsection{Tabelas}

As tabelas seguem o padrão ABNT, com bordas superiores e inferiores, conforme Tabela \ref{tab:exemplo}.

\begin{table}[h]
    \centering
    \caption{Exemplo de tabela formatada segundo ABNT}
    \label{tab:exemplo}
    \begin{tabular}{lcc}
        \hline
        \textbf{Item} & \textbf{Valor 1} & \textbf{Valor 2} \\
        \hline
        Linha 1 & 10,5 & 20,3 \\
        Linha 2 & 15,7 & 18,9 \\
        Linha 3 & 12,3 & 22,1 \\
        \hline
    \end{tabular}
    
    {\fontsize{10pt}{\baselineskip}\selectfont
   Fonte: O autor (2026)}
\end{table}

\subsubsection{Códigos}

Códigos-fonte podem ser inseridos diretamente ou importados de arquivos externos, como demonstrado no Código \ref{cod:python}.

\lstinputlisting[language=Python, caption=Exemplo de código Python, label=cod:python]{cod/exemplo.py}

Alternativamente, códigos curtos podem ser inseridos inline:

\begin{lstlisting}[language=Python, caption=Código inline]
def funcao_exemplo(x, y):
    return x + y

resultado = funcao_exemplo(5, 3)
print(f"Resultado: {resultado}")
\end{lstlisting}

\subsubsection{Níveis de Maturidade Tecnológica (TRL)}

A escala TRL (Technology Readiness Levels) é utilizada para avaliação de uma tecnologia de acordo com seu grau de desenvolvimento. Permite o acompanhamento de ativos tecnológicos durante os processos de pesquisa, desenvolvimento e validação, indicando o quão pronto se encontra um produto em sua escala de desenvolvimento.

\begin{table}[h]
    \centering
    \caption{Níveis de Maturidade Tecnológica (TRL)}
    \label{tab:trl}
    \small
    \begin{tabular}{p{1.5cm}p{2.5cm}p{7.5cm}}
        \hline
        \textbf{Nível} & \textbf{Definição Sintética} & \textbf{Descrição} \\
        \hline
        TRL 1 & Ideação & Princípios básicos observados e reportados \\
        \hline
        TRL 2 & Concepção & Concepção tecnológica e/ou aplicação formulada \\
        \hline
        TRL 3 & Prova de Conceito & Prova de conceitos das funções críticas de forma analítica ou experimental \\
        \hline
        TRL 4 & Otimização & Validação em ambiente de laboratório de componentes ou arranjos experimentais básicos \\
        \hline
        TRL 5 & Prototipagem & Validação em ambiente relevante de componentes ou arranjos experimentais com configurações físicas finais \\
        \hline
        TRL 6 & Escalonamento & Modelo do sistema/subsistema protótipo de demonstrador em ambiente relevante \\
        \hline
        TRL 7 & Demonstração em Ambiente Operacional & Protótipo do demonstrador do sistema em ambiente operacional \\
        \hline
        TRL 8 & Produção & Sistema completo, testado, qualificado e demonstrado \\
        \hline
        TRL 9 & Produção Continuada & Sistema já foi operado em todas as condições, extensão e alcance \\
        \hline
    \end{tabular}
    
    \vspace{0.3cm}
    {\fontsize{10pt}{\baselineskip}\selectfont
   Fonte: O autor (2026)}
\end{table}

% ==================================================
% CONCLUSÃO
% ==================================================
\section{Considerações Finais}

A conclusão deve retomar os objetivos apresentados na introdução e sintetizar os principais resultados obtidos no desenvolvimento do trabalho. É importante que seja concisa e objetiva, destacando as contribuições do estudo e possíveis trabalhos futuros.

Este modelo de relatório demonstrou a aplicação das normas \cite{abntex2013} ABNT para documentos técnicos, fornecendo exemplos práticos de estruturação, citações, figuras, tabelas e códigos-fonte.

% ==================================================
% REFERÊNCIAS
% ==================================================

\bibliographystyle{abntex2-alf}
\bibliography{referencias}

\end{document}
